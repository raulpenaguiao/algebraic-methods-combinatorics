\documentclass[kulak]{tplt} 
% options: kulak (default) or kul

\title{Algebraic Methods in Combinatorics}
\author{Assignment 3}
\date{Summer semester 2022 -- 2023}
\address{
	\textbf{Max Planck Institute for the Mathematics in the Sciences} \\
	\textbf{Universit\"at Leipzig}}


\usepackage{graphicx}
\usepackage{amssymb}
\usepackage{amsthm}
\usepackage{bbold}
\usepackage{listings}
\usepackage{lineno}
%\usepackage[margin=3cm]{geometry}
\usepackage[all,cmtip, color,matrix,arrow]{xy}
%\usepackage{amsaddr}
\usepackage{tikz-cd}
\usepackage{amsmath}%To use \text 
\usepackage[utf8]{inputenc}
\usepackage{hyperref}
\usepackage[capitalize]{cleveref}
\crefname{thm}{Theorem}{Theorems}
%\usepackage{bbold}
\usepackage[export]{adjustbox}
\usepackage{todonotes}
\usepackage{bm}
\usepackage{wrapfig}
\usepackage{float}
\usepackage{mathtools}
\usepackage{aliascnt}
\newaliascnt{eqfloat}{equation}
\newfloat{eqfloat}{h}{eqflts}
\floatname{eqfloat}{Equation}
\usepackage{dirtytalk}
\usepackage[mathscr]{euscript}


%\def\shuffle{\sqcup\mathchoice{\mkern-7mu}{\mkern-7mu}{\mkern-3.2mu}{\mkern-3.8mu}\sqcup\,}
\newcommand{\qshuffle}{\overline{\shuffle}}


\theoremstyle{definition}
\newtheorem{thm}{Theorem}[section]
\newtheorem{prop}[thm]{Proposition}
\newtheorem{lm}[thm]{Lemma}
\newtheorem{cor}[thm]{Corollary}
\newtheorem{obs}[thm]{Observation}
\newtheorem{defin}[thm]{Definition}
\newtheorem{smpl}[thm]{Example}
\newtheorem{quest}[thm]{Question}
\newtheorem{exe}[thm]{Exercise}
\newtheorem{const}[thm]{Construction}
\newtheorem{prob}[thm]{Problem}
\newtheorem{conj}[thm]{Conjecture}
\newtheorem{rem}[thm]{Remark}
\crefname{lm}{Lemma}{Lemmas}
\crefname{thm}{Theorem}{Theorems}
\crefname{prop}{Proposition}{Propositions}
\crefname{defin}{Definition}{Definitions}
\crefname{rem}{Remark}{Remarks}

\newcommand{\R}{\mathbb{R}}
\newcommand{\Z}{\mathbb{Z}}
\newcommand{\F}{\mathbb{F}}
\newcommand{\Q}{\mathbb{Q}}
\newcommand{\PP}{\mathbb{P}}

\newcommand{\one}{\mathbb{1}}

\newcommand{\CC}{\mathcal C}
\newcommand{\JJ}{\mathcal J}
\newcommand{\II}{\mathcal I}
\newcommand{\BB}{\mathcal B}
\newcommand{\FF}{\mathcal F}

%vectors
\newcommand{\vv}{\mathsf{v}}
\newcommand{\vw}{\mathsf{w}}
\newcommand{\vj}{\mathsf{j}}
\newcommand{\vx}{\mathsf{x}}
\newcommand{\vy}{\mathsf{y}}
\newcommand{\va}{\mathsf{a}}

\newcommand{\spn}{\mathrm{span}}
\newcommand{\rk}{\mathrm{rk}}
\newcommand{\tr}{\mathrm{tr}}
\newcommand{\conv}{\mathrm{conv}}
\newcommand{\maxcut}{\mathrm{maxcut}}
\newcommand{\Tr}{\mathrm{Tr}}
\newcommand{\we}{\mathrm{we}}


\begin{document}

\maketitle
\begin{enumerate}
\item 
Fix an integer $n$, so that $\chi_G(n)$ is the number of proper colourings of $G$ using $n$ colours.
Let $e = \{v_1, v_2\}$.
Fix $f$ a proper colouring of $G\setminus e$.
We have two distinct cases: either $f(v_1) \neq f(v_2)$, in which case $f$ is a proper colouring of the vertices of $G$, or $f(v_1) = f(v_2)$, in which case this induces $\hat{f}$ a colouring in $G/_e$ by setting $f(e) = f(v_1)$.
This is a proper colouring, and in fact any proper colouring of $G/_e$ can be obtained in such a way, we conclude that 
$$ \chi_{G\setminus e }(n) = \chi_G(n) + \chi_{G/_e}(n)\, . $$

\item 
\begin{enumerate}
\item 
Any proper colouring of $K_m$ has to use different colours for each vertex of the graph, so $\chi_{K_m}(n) = \frac{n!}{(n-m)!}$.

\item 
A proper colouring of the graph with no edges is simply a map $V(0_m) \to [n]$, so there are $\chi_{0_m}(n) = n^m$ many such colourings.


\item 
We claim that if a tree $T$ has $v$ vertices, then $\chi_T(n) = n(n-1)^v$.
We use induction on $v$.
If $v=1$, then the graph has no edges and so $\chi_T(n) = n^1 = n (n-1)^0$.

For the induction step, let $e$ be a leaf of the tree, so that $T\setminus e = \{\vv\} \uplus T'$ for some smaller tree $T$.
Therefore, $\chi_{T'\setminus e}(n) = n \chi_{T'}(n) = n^2(n-1)^{v-2}$, where we used the induction hypothesis on $T'$.
On the other hand, $T/_e$ is also a tree, so $\chi_{T/_e}(n) = n (n-1)^{v-2}$.
We conclude with the deletion contraction:
$$\chi_T(n) = \chi_{T'\setminus e}(n) - \chi_{T/_e}(n) = n^2(n-1)^{v-2}  - n(n-1)^{v-2} = n(n-1)^{v-1} \, .  $$


\item 
Let $C_m$ be the cycle on $m$ vertices, and $T_m$ a path on $m$ vertices.
If $m=3$ then $\chi_{C_3} = n(n-1)(n-2)$.
Notice that deleting an edge on $C_{m+1}$ we get $T_{m+1}$, which is a tree, and contracting gives us $C_m$.
Therefore, we use deletion contraction to get 
$$\chi_{C_{m+1}}(n) =  \chi_{T_{m+1}}(n) - \chi_{C_m}(n) =  n(n-1)^m - \chi_{C_m}(n) \, . $$

By summing this equation up to some $m$ we get, after using geometric series formulas, that
$$\chi_{C_m}(n) = (n-1)^2\left( (-1)^m - (n-1)^{m-2}\right) \, . $$

\end{enumerate}


\item 
Note that $A_{K_{m, n}}^2 = \begin{pmatrix}
n J & 0 \\ 0 & mJ
\end{pmatrix}$, where $J$ is the all-one matrix (either $m\times m$ or $n\times n$).
The spectrum of this matrix is $\{ 0 ^{\times (m+n-2) }, mn^{\times 2}\}$.
Indeed, any vector $\vv $ such that $\sum_{i=1}^m \vv_i = 0$ and $\sum_{i=m+1}^{m+n} \vv_i = 0$ has $A_{K_{m, n}}^2\vv =0 $ and the vectors $(\underbrace{1}_{m \text{ times}}, \underbrace{0}_{n \text{ times}})$ and  $ (\underbrace{0}_{m \text{ times}}, \underbrace{1}_{n \text{ times}})$ eigenvectors corresponding to the eigenvalue $mn$.

Thus, $0$ is an eigenvalue of $A_{K{m, n}}$ with multiplicity $m+n-2$ and the values $\{ \sqrt{mn}, -\sqrt{mn}\}$ have combined multiplicity two.
Because $\Tr A_{K_{m, n}} =0 $, we have that each of these values has multiplicity one.
Thus, the spectrum is 
$$ \{ \sqrt{mn}, \underbrace{0, 
\ldots, 0}_{m+n-2 \text{ many times }}, -\sqrt{mn} \} \, . $$

\item 
\begin{enumerate}
\item 

\item 

\item 

\item 

\end{enumerate}


\item 

\item 
\begin{enumerate}
\item 

\item 

\item 

\item

\item 
\end{enumerate}

\end{enumerate}
\end{document}
