\documentclass[kulak]{tplt} 
% options: kulak (default) or kul

\title{Algebraic Methods in Combinatorics}
\author{Assignment 2 - Solutions}
\date{Summer semester 2022 -- 2023}
\address{
	\textbf{Max Planck Institute for the Mathematics in the Sciences} \\
	\textbf{Universit\"at Leipzig}}


\usepackage{graphicx}
\usepackage{amssymb}
\usepackage{amsthm}
\usepackage{bbold}
\usepackage{listings}
\usepackage{lineno}
%\usepackage[margin=3cm]{geometry}
\usepackage[all,cmtip, color,matrix,arrow]{xy}
%\usepackage{amsaddr}
\usepackage{tikz-cd}
\usepackage{amsmath}%To use \text 
\usepackage[utf8]{inputenc}
\usepackage{hyperref}
\usepackage[capitalize]{cleveref}
\crefname{thm}{Theorem}{Theorems}
%\usepackage{bbold}
\usepackage[export]{adjustbox}
\usepackage{todonotes}
\usepackage{bm}
\usepackage{wrapfig}
\usepackage{float}
\usepackage{mathtools}
\usepackage{aliascnt}
\newaliascnt{eqfloat}{equation}
\newfloat{eqfloat}{h}{eqflts}
\floatname{eqfloat}{Equation}
\usepackage{dirtytalk}
\usepackage[mathscr]{euscript}


%\def\shuffle{\sqcup\mathchoice{\mkern-7mu}{\mkern-7mu}{\mkern-3.2mu}{\mkern-3.8mu}\sqcup\,}
\newcommand{\qshuffle}{\overline{\shuffle}}


\theoremstyle{definition}
\newtheorem{thm}{Theorem}[section]
\newtheorem{prop}[thm]{Proposition}
\newtheorem{lm}[thm]{Lemma}
\newtheorem{cor}[thm]{Corollary}
\newtheorem{obs}[thm]{Observation}
\newtheorem{defin}[thm]{Definition}
\newtheorem{smpl}[thm]{Example}
\newtheorem{quest}[thm]{Question}
\newtheorem{exe}[thm]{Exercise}
\newtheorem{const}[thm]{Construction}
\newtheorem{prob}[thm]{Problem}
\newtheorem{conj}[thm]{Conjecture}
\newtheorem{rem}[thm]{Remark}
\crefname{lm}{Lemma}{Lemmas}
\crefname{thm}{Theorem}{Theorems}
\crefname{prop}{Proposition}{Propositions}
\crefname{defin}{Definition}{Definitions}
\crefname{rem}{Remark}{Remarks}

\newcommand{\R}{\mathbb{R}}
\newcommand{\Z}{\mathbb{Z}}
\newcommand{\F}{\mathbb{F}}
\newcommand{\Q}{\mathbb{Q}}
\newcommand{\PP}{\mathbb{P}}

\newcommand{\one}{\mathbb{1}}

\newcommand{\CC}{\mathcal C}
\newcommand{\JJ}{\mathcal J}
\newcommand{\II}{\mathcal I}
\newcommand{\BB}{\mathcal B}
\newcommand{\FF}{\mathcal F}

%vectors
\newcommand{\vv}{\mathsf{v}}
\newcommand{\vw}{\mathsf{w}}
\newcommand{\vj}{\mathsf{j}}
\newcommand{\vx}{\mathsf{x}}
\newcommand{\vy}{\mathsf{y}}
\newcommand{\va}{\mathsf{a}}

\newcommand{\spn}{\mathrm{span}}
\newcommand{\rk}{\mathrm{rk}}
\newcommand{\tr}{\mathrm{tr}}
\newcommand{\conv}{\mathrm{conv}}
\newcommand{\maxcut}{\mathrm{maxcut}}
\newcommand{\Tr}{\mathrm{Tr}}
\newcommand{\we}{\mathrm{we}}
\newcommand{\Id}{\mathrm{Id}}
\newcommand{\spec}{\mathrm{spec}}


\begin{document}

\maketitle
\begin{enumerate}
\item 
Fix an integer $n$, so that $\chi_G(n)$ is the number of proper colourings of $G$ using $n$ colours.
Let $e = \{v_1, v_2\}$.
Fix $f$ a proper colouring of $G\setminus e$.
We have two distinct cases: either $f(v_1) \neq f(v_2)$, in which case $f$ is a proper colouring of $G$, or $f(v_1) = f(v_2)$, in which case this induces $\hat{f}$ a colouring in $G/_e$ by setting $f(e) = f(v_1)$.
This is a proper colouring, and in fact any proper colouring of $G/_e$ can be uniquely obtained in such a way.

This constructs a bijection between
\begin{align*}
 \{ \text{ proper colourings of $G$ with $n$ colours }\} &\uplus \{ \text{ proper colourings of $G/_e$ with $n$ colours }\} \\
 &\leftrightarrow \{ \text{ proper coloruings of $G\setminus e$ with $n$ colours }\}\, , 
\end{align*}
thus we conclude that 
$$ \chi_{G\setminus e }(n) = \chi_G(n) + \chi_{G/_e}(n)\, . $$

\item 
\begin{enumerate}
\item 
Any proper colouring of $K_m$ has to use different colours for each vertex of the graph, so $\chi_{K_m}(n) = n(n-1) \cdots (n-m+1) = \frac{n!}{(n-m)!}$.

\item 
A proper colouring of the graph with no edges is simply a map $V(0_m) \to [n]$, so there are $\chi_{0_m}(n) = n^m$ many such colourings.


\item 
We claim that if a tree $T$ has $v$ vertices, then $\chi_T(n) = n(n-1)^v$.
We use induction on $v$.
If $v=1$, then the graph has no edges and so $\chi_T(n) = n^1 = n (n-1)^0$.

For the induction step, let $e$ be a leaf of the tree, so that $T\setminus e$ is the disjoint union of two graphs, an isolated vertex $\{\vv\}$ and some smaller tree $T'$ with $v-1$ vertices.
Therefore, $\chi_{T \setminus e}(n) = n \chi_{T'}(n) = n^2(n-1)^{v-2}$, where we used the induction hypothesis on $T'$.
On the other hand, $T/_e$ is also a tree with $v-1$ vertices, so $\chi_{T/_e}(n) = n (n-1)^{v-2}$.
We conclude with the deletion contraction:
$$\chi_T(n) = \chi_{T'\setminus e}(n) - \chi_{T/_e}(n) = n^2(n-1)^{v-2}  - n(n-1)^{v-2} = n(n-1)^{v-1} \, .  $$


\item 
Let $C_m$ be the cycle on $m$ vertices, and $T_m$ a path on $m$ vertices.
Notice that deleting an edge on $C_{m+1}$ we get $T_{m+1}$, which is a tree, and that the contraction $C_{m+1}/_e$ is isomorphic to $C_m$.
Therefore, we use deletion contraction to get 
$$\chi_{C_{m+1}}(n) =  \chi_{T_{m+1}}(n) - \chi_{C_m}(n) =  n(n-1)^m - \chi_{C_m}(n) \, . $$

By multiplying this equation by $(-1)^m$ and summing for $m=3, \ldots, M$ we get,
\begin{align*}
\sum_{m=3}^M (-1)^m\chi_{C_{m+1}}(n) &= \sum_{m=3}^M (-1)^m n(n-1)^m - \sum_{m=3}^M (-1)^m \chi_{C_{m}}(n) \\
\sum_{m=3}^M (-1)^m(\chi_{C_{m+1}}(n) + \chi_{C_{m}}(n) ) &= n \frac{(n-1)^{M+1}(-1)^{M+1} - (n-1)^3(-1)^3}{-(n-1) -  1}\\
(-1)^M \chi_{C_{M+1}}(n) + (-1)^3 \chi_{C_{3}}(n) &= -(n-1)^{M+1}(-1)^{M+1} + (n-1)^3(-1)^3\\
\chi_{C_{M+1}}(n) + (-1)^{M+3} \chi_{C_{3}}(n) &= -(n-1)^{M+1}(-1)^{2M+1} + (n-1)^3(-1)^{M+3}\\
\chi_{C_{M+1}}(n) &= (n-1)^{M+1} + (n-1)^3(-1)^{M+3} - (-1)^{M+3} \chi_{C_{3}}(n)\\
\chi_{C_{M+1}}(n) &= (n-1)^{M+1} - (-1)^M(n-1)\left( (n-1)^2 - n(n-2) \right) \\
\chi_{C_{M+1}}(n) &= (n-1)^{M+1} - (-1)^M(n-1)
\end{align*}
where we use that $\chi_{C_{3}}(n) = n(n-1)(n-2)$, since $C_{3}$ is the complete graph on $3$ vertices.
We conclude the formula for $M\geq 3$ to be
$$\chi_{C_M}(n) = (n-1)^M + (-1)^M(n-1) \, . $$
\end{enumerate}


\item 
Recall that for any graph $G$, $(A_G^k)_{i, j}$ is the number of paths of length $k$ connecting the vertices $i$ and $j$.
Therefore, one can see that $A_{K_{m, n}}^2 = \begin{pmatrix}
n J_m & 0 \\ 0 & mJ_n
\end{pmatrix}$, where $J_k$ is the $k\times k$ all-one matrix.
The spectrum of this matrix is $\{ 0 ^{\times (m+n-2) }, mn^{\times 2}\}$.
Indeed, any vector $\vv $ such that $\sum_{i=1}^m \vv_i = 0$ and $\sum_{i=m+1}^{m+n} \vv_i = 0$ has $A_{K_{m, n}}^2\vv =0 $ and the vectors $(\underbrace{1}_{m \text{ times}}, \underbrace{0}_{n \text{ times}})$ and  $ (\underbrace{0}_{m \text{ times}}, \underbrace{1}_{n \text{ times}})$ are linearly independent eigenvectors corresponding to the eigenvalue $mn$.

Thus, $0$ is an eigenvalue of $A_{K{m, n}}$ with multiplicity $m+n-2$ and the eigenvalues $\{ \sqrt{mn}, -\sqrt{mn}\}$ have combined multiplicity two.
Because $\Tr A_{K_{m, n}} =0 $, we have that each of these values has multiplicity one.
Thus, the spectrum is 
$$ \{ \sqrt{mn}, \underbrace{0, \ldots, 0}_{m+n-2 \text{ many times }}, -\sqrt{mn} \} \, . $$

\item 
\begin{enumerate}
\item 
First we observe that $\vv_1 = \frac{1}{\sqrt{n}} \mathbb{1}$ is indeed an eigenvector.
Note that for each vertex $i$ we have $(A_G \vv_1)_i = \sum_{j : i\-- j} \frac{1}{\sqrt{n}} = \frac{d}{\sqrt{n}}$, as $i$ has exactly $d$ neighbours.

Assume now that there is some eigenvalue $\lambda > d$, and let $\vv $ be its corresponding eigenvector, let $i$ be the vertex such that $|\vv_i|$ is maximal.
Then 
\begin{align*}
\lambda |\vv_i | &= |(A_G \vv)_i|\\
 &= \left|\sum_{j : i\-- j} \vv \right| \\
 &\leq \sum_{j : i\-- j} |\vv| \\
 &\leq d |\vv_i| < \lambda |\vv_i | \, ,
\end{align*}
a contradiction.

\item 

\item 

\item 

\end{enumerate}


\item 
Let $P$ be the Petersen graph, and let $G$ be the line graph of $K_5$.
Because $K_5$ has $10$ edges, $G$ has $10$ vertices.
Furthermore, each edge in $K_5$ has $6$ neighbouring edges, so $G$ is $6$-regular.
One can notice that $P$ is the complementary graph of $G$, so we compute the eigenvalues of $G$ and use 4.d).

Let $B = B_{K_5}$ be the incidence matrix of $K_5$, that is for $v\in V(K_5)$ and $e\in E(K_5)$, the corresponding entry $B_{v, e}$ is $1$ if $v\in e$, and $0$ otherwise.
In this way we have 
\begin{equation}\label{eq:similar}
\begin{split}
A_{K_5} &= B B^T - 4 \Id_{5} \, , \\
A_G &= B^T B - 2 \Id_{10} \, .
\end{split}
\end{equation}

We will use \eqref{eq:similar} to compute the eigenvalues of $A_G$.
Specifically, we will show that any eigenvalue of $B B^T$ is an eigenvalue of $B^T B$, with at least the same multiplicity, by mapping an eigenspace to another injectively.
Finally, we will show that all remaning eigenvalues of $B^T B$ are zero, by analyzing its rank.
The connection between the spectrum og $G$ and $K_5$ is then done using \eqref{eq:similar}.

We start with the latter.
Recall that if $\mu_{\lambda}$ is the multiplicity of the eigenvalue $\lambda $ in the matrix $A$, $\rk(A -\lambda  \Id_n ) + \mu_{\lambda} \geq n$ with equality for symetric matrices (from the spectral theorem).
It is easy to see that $\rk B^T = 5 $ is maximal: if $e_1, e_2, e_3$ are three edges that form a triangle, then $B_{\cdot, e_1} + B_{\cdot, e_2} - B_{\cdot, e_3}$ is a multiple of a canonical basis element.
Thus, we observe that $\rk(A_G + 2 \Id_{10} ) = \rk (B^T B) \leq \rk B^T = 5$.
Therefore $\mu_{-2} \geq 5$.

Now consider the eigenspace $V_{\lambda} $ of $B B^T$ corresponding to the eigenvalue $\lambda$, and the eigenspace $W_{\lambda} $ of $B^T B$ corresponding to the eigenvalue $\lambda$.
The map $B^T$ is an injective map $\R^5 \to \R^{10}$, as $\rk B^T = 5$.
Furthermore, if $\vv \in V_{\lambda} $, we have $B^T B B^T v = \lambda B^T v$.
So $B^T V_{\lambda} \subseteq W_{\lambda}$.


We now compute the spectrum of each of the matrices.
First, we know that 
$$\spec \, K_5 = \{4^{\times 1 }, -1^ {\times 4} \}\, ,$$
thus
$$\spec \, B B^T = \{8^{\times 1 }, 3^{\times 4} \}\, ,$$
and therefore the eigenspace of $B^T B$ satisfies
$$\spec \, B B^T \subseteq \{8^{\times 1 }, 3^{\times 4}, 0^{\times 5} \}\, ,$$
and since this already has the correct cardinality $10$, this is precisely the spectrum of the matrix.

Thus
$$\spec \, G = \{6^{\times 1 }, 1^{\times 4}, -2^ {\times 5} \}\, ,$$

Therefore, from 4.d)
$$\spec \, P = \{3^{\times 1}, 1^{\times 5}, -2^{\times 4} \}\, .$$
Alternatively, the following python code, yields the spectrum of $A_P$:
\begin{verbatim}

\end{verbatim}

\item 
\begin{enumerate}
\item 

\item 

\item 

\item

\item 
From 6.a), we have that $\dim (V^i \cap (V^{i+1})^{\perp}) = \binom{e}{i}-\binom{e}{i-1}$.
From 6.d) we have that $(V^i \cap (V^{i+1})^{\perp}) \subseteq V_{\lambda_i}$ is contained in the eigenspace of $A_{K(E, r)}$ corresponding to $(-1)^i\binom{e-r-i}{r-i} = \lambda_i$.

We have $\binom{e}{r} \geq \sum_{i=0}^r \dim V_{\lambda_i} \geq \sum_{i=0}^r \binom{e}{i}-\binom{e}{i-1} = \binom{e}{r}$.
So we have equality all throughout, so $(V^i \cap (V^{i+1})^{\perp}) = V_{\lambda_i}$ and these are indeed all the eigenvalues of $K(E, r)$, so
$$\spec \, K(E, r) = \{  \lambda_i^{\binom{e}{i} - \binom{e}{i-1}} | i = 0, \ldots, r\} \, . $$
\end{enumerate}

\end{enumerate}
\end{document}
