\documentclass[kulak]{tplt} 
% options: kulak (default) or kul

\title{Algebraic Methods in Combinatorics}
\author{Assignment 1}
\date{Summer semester 2022 -- 2023}
\address{
	\textbf{Max Planck Institute for the Mathematics in the Sciences} \\
	\textbf{Universit\"at Leipzig}}


\usepackage{graphicx}
\usepackage{amssymb}
\usepackage{amsthm}
\usepackage{bbold}
\usepackage{listings}
\usepackage{lineno}
%\usepackage[margin=3cm]{geometry}
\usepackage[all,cmtip, color,matrix,arrow]{xy}
%\usepackage{amsaddr}
\usepackage{tikz-cd}
\usepackage{amsmath}%To use \text 
\usepackage[utf8]{inputenc}
\usepackage{hyperref}
\usepackage[capitalize]{cleveref}
\crefname{thm}{Theorem}{Theorems}
%\usepackage{bbold}
\usepackage[export]{adjustbox}
\usepackage{todonotes}
\usepackage{bm}
\usepackage{wrapfig}
\usepackage{float}
\usepackage{mathtools}
\usepackage{aliascnt}
\newaliascnt{eqfloat}{equation}
\newfloat{eqfloat}{h}{eqflts}
\floatname{eqfloat}{Equation}
\usepackage{dirtytalk}
\usepackage[mathscr]{euscript}


%\def\shuffle{\sqcup\mathchoice{\mkern-7mu}{\mkern-7mu}{\mkern-3.2mu}{\mkern-3.8mu}\sqcup\,}
\newcommand{\qshuffle}{\overline{\shuffle}}


\theoremstyle{definition}
\newtheorem{thm}{Theorem}[section]
\newtheorem{prop}[thm]{Proposition}
\newtheorem{lm}[thm]{Lemma}
\newtheorem{cor}[thm]{Corollary}
\newtheorem{obs}[thm]{Observation}
\newtheorem{defin}[thm]{Definition}
\newtheorem{smpl}[thm]{Example}
\newtheorem{quest}[thm]{Question}
\newtheorem{exe}[thm]{Exercise}
\newtheorem{const}[thm]{Construction}
\newtheorem{prob}[thm]{Problem}
\newtheorem{conj}[thm]{Conjecture}
\newtheorem{rem}[thm]{Remark}
\crefname{lm}{Lemma}{Lemmas}
\crefname{thm}{Theorem}{Theorems}
\crefname{prop}{Proposition}{Propositions}
\crefname{defin}{Definition}{Definitions}
\crefname{rem}{Remark}{Remarks}

\newcommand{\R}{\mathbb{R}}
\newcommand{\Z}{\mathbb{Z}}
\newcommand{\F}{\mathbb{F}}
\newcommand{\Q}{\mathbb{Q}}
\newcommand{\PP}{\mathbb{P}}

\newcommand{\one}{\mathbb{1}}

\newcommand{\CC}{\mathcal C}
\newcommand{\JJ}{\mathcal J}
\newcommand{\II}{\mathcal I}
\newcommand{\BB}{\mathcal B}
\newcommand{\FF}{\mathcal F}

%vectors
\newcommand{\vv}{\mathsf{v}}
\newcommand{\vw}{\mathsf{w}}
\newcommand{\vj}{\mathsf{j}}
\newcommand{\vx}{\mathsf{x}}
\newcommand{\vy}{\mathsf{y}}
\newcommand{\va}{\mathsf{a}}

\newcommand{\spn}{\mathrm{span}}
\newcommand{\rk}{\mathrm{rk}}
\newcommand{\tr}{\mathrm{tr}}
\newcommand{\conv}{\mathrm{conv}}
\newcommand{\maxcut}{\mathrm{maxcut}}
\newcommand{\we}{\mathrm{we}}


\begin{document}

\maketitle
\begin{enumerate}
\item Generalised Fischer Inequality

Let $E = \{1, 2, 3, 4, 5\}$, $L =\{0, 2\}$ and $\FF $ is the $L$-Fischer family $\FF = \{1234, 125, 235\}$.

Compute $f_F$ for each $F\in \FF$, as described in the proof of the generalised Fischer inequality.


\item Non-prime towns.

Let $E$ be a finite set, and $s$ a positive integer.
A family $\FF \subseteq 2^E$ is called an $s$-town if 
\begin{itemize}
\item For any $F \in \FF$ we have that $|F|$ is not a multiple of $s$.

\item For distinct $F,  G \in \FF$, we have that $|F\cap G|$ is a multiple of $s$.
\end{itemize}

\begin{enumerate}
\item Show that if $s$ is a power of a prime, then $|\FF| \leq |E|$.
(Hint: Use the field $\Q$)

\item Show that if $s = 6$, then $|\FF| \leq 2 |E|$.

\item Show that there exists a constant $c = c(s)$ that depends on $s$ such that for any $s$-town we have $|\FF| \leq c(s) |E|$.
\end{enumerate}



\item Reverse club problem

Let $E$ be a finite set, and $\FF \subseteq 2^E$ such that $|F|$ is even for any $F\in \FF$, and $|F\cap G|$ is odd for distinct $F, G \in \FF$.


\begin{enumerate}

\item Show that no more than $|E|$ clubs are possible. 
Show that if $|E|$ is odd, this is tight.

\item Show that if $|E|$ is even, only $|E| - 1$ clubs are possible.
\end{enumerate}

\item Two oddtowns

Let $E$ be a finite set, and $\BB, \CC\subseteq 2^E$ such that $|B\cap C|$ is odd for sets $B\in \BB, \, C \in \CC$.

Show that $|\BB| |\CC| \leq 2^{n-1}$.


\item Colourings and coherent families

Let $E$ be a finite set.
A collection $\FF \subseteq 2^E $ is said to be \textbf{two colourable} if there is a colouring of $E$ using two colours in such a way that there is no monochrmatic set $F \in \FF$.

Take a collection $\BB \subseteq E$ such that $\bigcup_{B \in \BB} B = E$ that is not two colourable, but any subfamily of $\BB$ is two colourable.
Prove that $|\BB| \geq |E|$.



\end{enumerate}
\end{document}
