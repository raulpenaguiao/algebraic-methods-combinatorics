\documentclass[kulak]{tplt} 
% options: kulak (default) or kul

\title{Algebraic Methods in Combinatorics}
\author{Assignment 4}
\date{Summer semester 2022 -- 2023}
\address{
	\textbf{Max Planck Institute for the Mathematics in the Sciences} \\
	\textbf{Universit\"at Leipzig}}


\usepackage{graphicx}
\usepackage{amssymb}
\usepackage{amsthm}
\usepackage{bbold}
\usepackage{listings}
\usepackage{lineno}
%\usepackage[margin=3cm]{geometry}
\usepackage[all,cmtip, color,matrix,arrow]{xy}
%\usepackage{amsaddr}
\usepackage{tikz-cd}
\usepackage{amsmath}%To use \text 
\usepackage[utf8]{inputenc}
\usepackage{hyperref}
\usepackage[capitalize]{cleveref}
\crefname{thm}{Theorem}{Theorems}
%\usepackage{bbold}
\usepackage[export]{adjustbox}
\usepackage{todonotes}
\usepackage{bm}
\usepackage{wrapfig}
\usepackage{float}
\usepackage{mathtools}
\usepackage{aliascnt}
\newaliascnt{eqfloat}{equation}
\newfloat{eqfloat}{h}{eqflts}
\floatname{eqfloat}{Equation}
\usepackage{dirtytalk}
\usepackage[mathscr]{euscript}


%\def\shuffle{\sqcup\mathchoice{\mkern-7mu}{\mkern-7mu}{\mkern-3.2mu}{\mkern-3.8mu}\sqcup\,}
\newcommand{\qshuffle}{\overline{\shuffle}}


\theoremstyle{definition}
\newtheorem{thm}{Theorem}[section]
\newtheorem{prop}[thm]{Proposition}
\newtheorem{lm}[thm]{Lemma}
\newtheorem{cor}[thm]{Corollary}
\newtheorem{obs}[thm]{Observation}
\newtheorem{defin}[thm]{Definition}
\newtheorem{smpl}[thm]{Example}
\newtheorem{quest}[thm]{Question}
\newtheorem{exe}[thm]{Exercise}
\newtheorem{const}[thm]{Construction}
\newtheorem{prob}[thm]{Problem}
\newtheorem{conj}[thm]{Conjecture}
\newtheorem{rem}[thm]{Remark}
\crefname{lm}{Lemma}{Lemmas}
\crefname{thm}{Theorem}{Theorems}
\crefname{prop}{Proposition}{Propositions}
\crefname{defin}{Definition}{Definitions}
\crefname{rem}{Remark}{Remarks}

\newcommand{\R}{\mathbb{R}}
\newcommand{\Z}{\mathbb{Z}}
\newcommand{\F}{\mathbb{F}}
\newcommand{\Q}{\mathbb{Q}}
\newcommand{\PP}{\mathbb{P}}

\newcommand{\one}{\mathbb{1}}

\newcommand{\CC}{\mathcal C}
\newcommand{\JJ}{\mathcal J}
\newcommand{\II}{\mathcal I}
\newcommand{\BB}{\mathcal B}
\newcommand{\FF}{\mathcal F}

%vectors
\newcommand{\vv}{\mathsf{v}}
\newcommand{\vw}{\mathsf{w}}
\newcommand{\vj}{\mathsf{j}}
\newcommand{\vx}{\mathsf{x}}
\newcommand{\vy}{\mathsf{y}}
\newcommand{\va}{\mathsf{a}}

\newcommand{\spn}{\mathrm{span}}
\newcommand{\rk}{\mathrm{rk}}
\newcommand{\tr}{\mathrm{tr}}
\newcommand{\conv}{\mathrm{conv}}
\newcommand{\maxcut}{\mathrm{maxcut}}
\newcommand{\we}{\mathrm{we}}


\begin{document}

\maketitle
\vspace{2mm}
\begin{enumerate}
\item Kemnitz conjecture.

For an additive group $G$ and an integer $n$, we write $f(G, n)$ for the smallest integer $f$ such that for any choice $g_1, \ldots, g_f \in G$ we can find $I\subseteq [f]$ with $|I| = n$ and $\sum_{i\in I} g_i = 0$.
The celebrated Erd\"os-Ginzburg-Ziv theorem shows that $f(\Z_n, n) = 2n -1$, and Kemnitz conectured that $f(\Z_n^2, n) = 4n-3$.

Recall that for sets $I, J\subseteq \Z_m^2$ and an integer $x\in\Z_{\geq 0}$, we write $\sum I $ for $\sum_{i\in I} i$, and also write
$$ (x | J) \coloneqq \# \left\{ I \subseteq J \Big| \sum I = 0, \, |I| = x \right\} \in \left\{ 0, 1, \ldots, \binom{|J|}{x} \right\}\, . $$

\begin{enumerate}
\item Show that $f(\Z_n^2, n) \geq 4n-3$, by finding a suitable sequence $g_1, \ldots, g_{4n-4}$ of elements in $\Z_n^2$.

\item Show that $f(\Z^2_2, 2) = 5$.

\item Show that $A = \{m\in \Z_{\geq 0} | f(\Z^2_2, 2) = 4m - 3 \} $ is a multiplicative set.
That is $a, b\in A \Rightarrow ab \in A$.

\item Show that for $p$ prime and a multiset $J \subseteq \Z_p^2$, we have the following:
\begin{align*}
|J| = 3p - 3 &\Rightarrow 1 - (p-1|J) - (p|J) + (2p-1|J) +(2p|J)  = 0 \text{ mod $p$.} \\
|J| = 3p - 2 \text{ or } |J| = 3p - 1 &\Rightarrow 1 - (p|J) + (2p|J)  = 0 \text{ mod $p$.} \\
|J| = 4p - 3 &\Rightarrow 1 - (p|J) + (2p|J) - (3p|J)  = 0 \text{ mod $p$.} \\
|J| = 4p - 3 &\Rightarrow (p-1|J) - (2p-1|J) + (3p-1|J)  = 0 \text{ mod $p$.}
\end{align*}
\end{enumerate}


\item Hypergraph union of edges.

\begin{enumerate}
\item Let $n, d, p$ be integers such that $p$ is prime and $n > d(p-1)$.
Show that any polynomial $h \in \F_p[\vx_1, \ldots, \vx_n]$ with $\deg h \leq d$ and $h(0, \ldots, 0) = 0$, then there exists a non-zero vector $\vv \in \{0, 1 \}^n$ such that $h(e) = 0$.

\item 
Recall that a hypergraph $H = (V, E)$ is a pair where $E \subseteq 2^V$, that is edges are allowed to have any number of endpoints.
The degree of a vertex in $H$ is the numbero of incident edges $\deg_H (v) = \sum_{\substack{e\in E \\ v \in e}} 1$.

Show that if $H$ has more than $d(p-1)$ edges and maximal degree at most $d$, then it contains a subset of edges whose union $U$ is such that $|U|$ is a multiple of $p$.

\end{enumerate}

\item Good orders of subsets of $\Z_p$.

Let $k$ be an integer and $p$ be an odd prime, and let $A, B \subseteq \Z_p$ sets of size $k$.
Prove that there is are orderings $(a_1, \ldots, a_k)$ of $A$ and $(b_1, \ldots, b_k)$ of $B$ such that $\{a_i+b_i\}_{i=1}^k$ are all distinct sums.

Hint: Fix an ordering for $A$ and find a polynmial that vanishes when a sequence of $b_i$ is bad.

\item Generalisation of Erd\"os-Heilbronn.

Let $p$ be a prime number and $A_0, \ldots, A_k \subseteq \Z_p$.
Assume that $|A_i|\neq |A_j| $ for $i\neq j$, and that $A_i \neq \emptyset $ for all $i$.

Then 
$$ \#\left\{ a_0 + \cdots + a_k \Big| a_i \neq a_j \forall_{i\neq j} \, , a_i \in A_i \right\} \geq \min \left\{p, \left( \sum_{i=0}^k |A_i| \right) - \binom{k+2}{2}  + 1 \right\}\, . $$

Hint: $\prod_{i < j} (x_i - x_j) = \det \begin{pmatrix}
1 & 1 & \cdots & 1\\
x_0 & x_1 & \cdots & x_k \\
\vdots & \vdots & \ddots & \vdots \\
x_0^k & x_1^k & \cdots & x_k^k 
\end{pmatrix}= \det \begin{pmatrix}
\frac{x_0!}{x_0!} & \frac{x_1!}{x_1!} & \cdots & \frac{x_k!}{x_k!}\\
\frac{x_0!}{(x_0-1)!} & \frac{x_1!}{(x_1-1)!} & \cdots & \frac{x_k!}{(x_k-1)!} \\
\vdots & \vdots & \ddots & \vdots \\
\frac{x_0!}{(x_0-k)!} & \frac{x_1!}{(x_1-k)!} & \cdots & \frac{x_k!}{(x_k-k)!}
\end{pmatrix}$

\end{enumerate}
\end{document}
