\documentclass[kulak]{tplt} 
% options: kulak (default) or kul

\title{Algebraic Methods in Combinatorics}
\author{Assignment 2}
\date{Summer semester 2022 -- 2023}
\address{
	\textbf{Max Planck Institute for the Mathematics in the Sciences} \\
	\textbf{Universit\"at Leipzig}}


\usepackage{graphicx}
\usepackage{amssymb}
\usepackage{amsthm}
\usepackage{bbold}
\usepackage{listings}
\usepackage{lineno}
%\usepackage[margin=3cm]{geometry}
\usepackage[all,cmtip, color,matrix,arrow]{xy}
%\usepackage{amsaddr}
\usepackage{tikz-cd}
\usepackage{amsmath}%To use \text 
\usepackage[utf8]{inputenc}
\usepackage{hyperref}
\usepackage[capitalize]{cleveref}
\crefname{thm}{Theorem}{Theorems}
%\usepackage{bbold}
\usepackage[export]{adjustbox}
\usepackage{todonotes}
\usepackage{bm}
\usepackage{wrapfig}
\usepackage{float}
\usepackage{mathtools}
\usepackage{aliascnt}
\newaliascnt{eqfloat}{equation}
\newfloat{eqfloat}{h}{eqflts}
\floatname{eqfloat}{Equation}
\usepackage{dirtytalk}
\usepackage[mathscr]{euscript}


%\def\shuffle{\sqcup\mathchoice{\mkern-7mu}{\mkern-7mu}{\mkern-3.2mu}{\mkern-3.8mu}\sqcup\,}
\newcommand{\qshuffle}{\overline{\shuffle}}


\theoremstyle{definition}
\newtheorem{thm}{Theorem}[section]
\newtheorem{prop}[thm]{Proposition}
\newtheorem{lm}[thm]{Lemma}
\newtheorem{cor}[thm]{Corollary}
\newtheorem{obs}[thm]{Observation}
\newtheorem{defin}[thm]{Definition}
\newtheorem{smpl}[thm]{Example}
\newtheorem{quest}[thm]{Question}
\newtheorem{exe}[thm]{Exercise}
\newtheorem{const}[thm]{Construction}
\newtheorem{prob}[thm]{Problem}
\newtheorem{conj}[thm]{Conjecture}
\newtheorem{rem}[thm]{Remark}
\crefname{lm}{Lemma}{Lemmas}
\crefname{thm}{Theorem}{Theorems}
\crefname{prop}{Proposition}{Propositions}
\crefname{defin}{Definition}{Definitions}
\crefname{rem}{Remark}{Remarks}

\newcommand{\R}{\mathbb{R}}
\newcommand{\Z}{\mathbb{Z}}
\newcommand{\F}{\mathbb{F}}
\newcommand{\Q}{\mathbb{Q}}
\newcommand{\PP}{\mathbb{P}}

\newcommand{\one}{\mathbb{1}}

\newcommand{\CC}{\mathcal C}
\newcommand{\JJ}{\mathcal J}
\newcommand{\II}{\mathcal I}
\newcommand{\BB}{\mathcal B}
\newcommand{\FF}{\mathcal F}

%vectors
\newcommand{\vv}{\mathsf{v}}
\newcommand{\vw}{\mathsf{w}}
\newcommand{\vj}{\mathsf{j}}
\newcommand{\vx}{\mathsf{x}}
\newcommand{\vy}{\mathsf{y}}
\newcommand{\va}{\mathsf{a}}
\newcommand{\vb}{\mathsf{b}}
\newcommand{\vc}{\mathsf{c}}
\newcommand{\ve}{\mathsf{e}}

\newcommand{\spn}{\mathrm{span}}
\newcommand{\rowspn}{\mathrm{rowspan}}
\newcommand{\rk}{\mathrm{rk}}
\newcommand{\tr}{\mathrm{tr}}
\newcommand{\conv}{\mathrm{conv}}
\newcommand{\maxcut}{\mathrm{maxcut}}
\newcommand{\we}{\mathrm{we}}
\newcommand{\Id}{\mathrm{Id}}
\newcommand{\spec}{\mathrm{spec}}


\begin{document}

\maketitle
\vspace{2mm}
\begin{enumerate}
\item Deletion contraction.

Prove the deletion contraction relation on the chromatic polynomial of a graph.
That is, let $G$ be a graph and $e$ any edge in $G$.
Denote by $G\setminus e $ the deletion of the edge $e$ in $G$, and $G /_e$ the contraction of the edge $e$ in $G$ (defined in the lecture notes).
Show that
$$ \chi_G(n) = \chi_{G\setminus e}(n) - \chi_{G /_e}(n)\, . $$

\item Chromatic polynomial of some graphs.

\begin{enumerate}
\item Compute the chromatic polynomial of a complete graph on $m$ vertices $K_m$.

\item Compute the chromatic polynomial of a graph with no edges on $m$ vertices $\mathbb{0}_m$.

\item Compute the chromatic polynomial of any tree graph (that is, connected and acyclic graph) on $m$ vertices.

\item Compute the chromatic polynomial of a cycle of length $m$.
Hint: Use deletion contraction.
\end{enumerate}



\item Spectrum of bipartite graphs.

Compute the spectrum of a complete bipartite graph $K_{m, n}$.
Hint: Compute $A_{K_{m, n}}^2$.



\item Helpful theorems on the spectrum of graphs.

Write as usual $\spec \, G = \{\lambda_1 \leq \cdots \leq \lambda_n \}$, and consider the corresponding orthonormal eigenbasis $\{\vv_1, \ldots, \vv_n\}$.


\begin{enumerate}
\item If $G$ is a $d$-regular graph, then $\lambda_1 = d$ and $\vv_1 = \frac{1}{\sqrt{n}}\mathbb{1}$.

\item If $G$ is connected, then $\lambda_1 \neq \lambda_2$. Hint: Recall Perron-Frobenius theorem.

\item A graph $G$ is bipartite if and only if $\spec \, G = - \spec \, G$.
Hint: recall that a graph is bipartite if and only if it does not have any cycle with odd length.

\item If a graph $G$ is $d$-regular, and $\overline{G}$ is the complementary graph, then $\spec \, \overline{G} = \{ \overline{\lambda_{n+1}} \geq  \overline{\lambda_{n}} \geq  \cdots \geq  \overline{\lambda_{3}} \geq  \overline{\lambda_{2}} \}$, where $\overline{\lambda_{n+1}} = n - d - 1$ and $\overline{\lambda_i} = - 1 - \lambda _ i$ for $i = 2, \ldots, n$.
Show that the corresponding eigenbasis is the same.
\end{enumerate}


\item Spectrum of the Petersen graph.

Compute the spectrum of the Petersen graph.
For that, take advantage of the previous exercise, and note that the complement of the Petersen graph is the line graph of $K_5$.

\item Spectrum of the Kneser graph.

In this exercise we will compute the spectrum of the Kneser graph.
Recall that the Kneser graph $K(E, r)$ is the graph with vertex set $V = \binom{E}{r}$ and two vertices $A, B$ share an edge if $A \cap B = \emptyset$.
We work on the vector space $\R^V$, where vectors are indexed by subsets of $E$ with $r$ elements.

For $i, j = 0, \ldots, r$, define the matrices $M_{i, j} = [m_{I, J}]_{\substack{I\in \binom{E}{i} \\ J \in \binom{E}{j}}}$ and $N_{i, j} = [n_{I, J}]_{\substack{I\in \binom{E}{i} \\ J \in \binom{E}{j}}}$ such that 

$$ m_{I, J} =\begin{cases*}
      & 1 \text{ if $I \subseteq J$,}\\
      & 0 \text{ otherwise,}
    \end{cases*}  \quad \quad \quad 
     n_{I, J} =\begin{cases*}
      & 1 \text{ if $I \cap J = \emptyset $,}\\
      & 0 \text{ otherwise.}
    \end{cases*}  $$

Note that $A_{K(E, r)} = N_{r, r}$.
Define $V^i = \rowspn \, M_{i, r} \subseteq \R^V$.
Let $e = |E|$.

\begin{enumerate}
\item
Show the following equalities:
\begin{equation}\label{eq:matrices}
\begin{split}
M_{i, j} M_{j, k} = \binom{k-i}{j-i} M_{i, k} &\text{ for $0\leq i , j, k \leq r$}\\
M_{i, j} N_{j, k} = \binom{e-k-i}{j-i} N_{i, k}  &\text{ for $0\leq i \leq j \leq k \leq e - r$}\\
\sum_{i=0}^j (-1)^i M_{i, j}^T N_{i, k} = M_{j, k} &\text{ for $0\leq j \leq k \leq r$}\\
\sum_{i=0}^j (-1)^i M_{i, j}^T M_{i, k} = N_{j, k} &\text{ for $0\leq j \leq k \leq r$}\\
\end{split}
\end{equation}

\item
Prove that $\dim V^i = \binom{e}{i}$ for $i=0, \ldots, r$. 
Hint: Show that $M_{i, r} \left( \sum_{j=0}^i \frac{(-1)^j}{\binom{e-i-j}{r-i}} N_{r, j} M_{j, i}\right) = \Id$.

\item Show that $V^i \subseteq V^{i+1}$ for $i=0, \ldots, r-1$.
Hint: Show that if $A = BC $ then $\rowspn\,~A~\subseteq~\rowspn\,~C$.

\item Show that $\rowspn \, N_{i, r} = \rowspn \, M_{i, r} = V^i$.
Hint: Use the 3rd and 4th equation in \eqref{eq:matrices}.

\item Show that if $\vv \in V^i \cap (V^{i-1})^{\perp}$, then $(-1)^i \vv A_G = \binom{e-r-i}{r-i} \vv $.


\item Conclude with the spectrum of $K(E, r)$.
\end{enumerate}
\end{enumerate}
\end{document}
