\documentclass[kulak]{tplt} 
% options: kulak (default) or kul

\title{Algebraic Methods in Combinatorics}
\author{Assignment 1}
\date{Summer semester 2022 -- 2023}
\address{
	\textbf{Max Planck Institute for the Mathematics in the Sciences} \\
	\textbf{Universit\"at Leipzig}}


\usepackage{graphicx}
\usepackage{amssymb}
\usepackage{amsthm}
\usepackage{bbold}
\usepackage{listings}
\usepackage{lineno}
%\usepackage[margin=3cm]{geometry}
\usepackage[all,cmtip, color,matrix,arrow]{xy}
%\usepackage{amsaddr}
\usepackage{tikz-cd}
\usepackage{amsmath}%To use \text 
\usepackage[utf8]{inputenc}
\usepackage{hyperref}
\usepackage[capitalize]{cleveref}
\crefname{thm}{Theorem}{Theorems}
%\usepackage{bbold}
\usepackage[export]{adjustbox}
\usepackage{todonotes}
\usepackage{bm}
\usepackage{wrapfig}
\usepackage{float}
\usepackage{mathtools}
\usepackage{aliascnt}
\newaliascnt{eqfloat}{equation}
\newfloat{eqfloat}{h}{eqflts}
\floatname{eqfloat}{Equation}
\usepackage{dirtytalk}
\usepackage[mathscr]{euscript}


%\def\shuffle{\sqcup\mathchoice{\mkern-7mu}{\mkern-7mu}{\mkern-3.2mu}{\mkern-3.8mu}\sqcup\,}
\newcommand{\qshuffle}{\overline{\shuffle}}


\theoremstyle{definition}
\newtheorem{thm}{Theorem}[section]
\newtheorem{prop}[thm]{Proposition}
\newtheorem{lm}[thm]{Lemma}
\newtheorem{cor}[thm]{Corollary}
\newtheorem{obs}[thm]{Observation}
\newtheorem{defin}[thm]{Definition}
\newtheorem{smpl}[thm]{Example}
\newtheorem{quest}[thm]{Question}
\newtheorem{exe}[thm]{Exercise}
\newtheorem{const}[thm]{Construction}
\newtheorem{prob}[thm]{Problem}
\newtheorem{conj}[thm]{Conjecture}
\newtheorem{rem}[thm]{Remark}
\crefname{lm}{Lemma}{Lemmas}
\crefname{thm}{Theorem}{Theorems}
\crefname{prop}{Proposition}{Propositions}
\crefname{defin}{Definition}{Definitions}
\crefname{rem}{Remark}{Remarks}

\newcommand{\R}{\mathbb{R}}
\newcommand{\Z}{\mathbb{Z}}
\newcommand{\F}{\mathbb{F}}
\newcommand{\Q}{\mathbb{Q}}
\newcommand{\PP}{\mathbb{P}}

\newcommand{\one}{\mathbb{1}}

\newcommand{\CC}{\mathcal C}
\newcommand{\JJ}{\mathcal J}
\newcommand{\II}{\mathcal I}
\newcommand{\BB}{\mathcal B}
\newcommand{\FF}{\mathcal F}

%vectors
\newcommand{\vv}{\mathsf{v}}
\newcommand{\vw}{\mathsf{w}}
\newcommand{\vj}{\mathsf{j}}
\newcommand{\vx}{\mathsf{x}}
\newcommand{\vy}{\mathsf{y}}
\newcommand{\va}{\mathsf{a}}

\newcommand{\spn}{\mathrm{span}}
\newcommand{\rk}{\mathrm{rk}}
\newcommand{\tr}{\mathrm{tr}}
\newcommand{\conv}{\mathrm{conv}}
\newcommand{\maxcut}{\mathrm{maxcut}}
\newcommand{\we}{\mathrm{we}}
\newcommand{\nul}{\mathrm{nul}}
%\newcommand{\ker}{\mathrm{ker}}


\begin{document}

\maketitle
\begin{enumerate}
\item Generalised Fischer inequality

First we note that $\FF $ is indeed an $L$-Fischer family, as $1234 \cap 125 = 12 $ has size $2$, $1234 \cap 235 = 23 $ has size $2$, and $125 \cap 235 = 25 $ also has size $2$.

We order the sets in $\FF $: $F_1 = 125$, $F_2 = 235$ and $F_3 = 1234$.
This ensures that $|F_1 | \leq |F_2| \leq |F_3|$.
In this way recall that
$$f_i (\vx) = \prod_{\substack{\ell \in L \\ \ell < |F_i|}} (\vx \cdot \vv_{F_i} - \ell ) \, . $$

Thus we have 
$$f_1(\vx) = ( \vx \cdot \vv_{F_1} ) (\vx \cdot \vv_{F_1} - 2 ) = (x_1 + x_2 + x_5 ) (x_1 + x_2 + x_5 - 2) \,  . $$
$$f_2(\vx) = ( \vx \cdot \vv_{F_2} ) (\vx \cdot \vv_{F_2} - 2 ) = (x_1 + x_3 + x_5 ) (x_1 + x_3 + x_5 - 2) \,  . $$
$$f_3(\vx) = ( \vx \cdot \vv_{F_3} ) (\vx \cdot \vv_{F_3} - 2 ) = (x_1 + x_2 + x_3 + x_4 ) (x_1 + x_2 + x_3 + x_4 - 2) \,  . $$


\item Non-prime towns.

\begin{enumerate}
\item 
Let $s = p^a$, where $p$ is a prime number.
As is usual, for each set $I \subseteq E$, define the vector $\vv_I \in \Q^E$:
$$ (\vv_I)_i =\begin{cases*}
      & 1 \text{ if $i \in I$,}\\
      & 0 \text{ otherwise.}
    \end{cases*} $$

We will show that $\{\vv_F\}_{F\in \FF}$ is linearly independent.
Assume that $\sum_{F \in \FF} \alpha_F \vv_F = \vec{0}$, where each $\alpha_F \in \Q$.
Assume that at least one $\alpha_F$ is non-zero.
By multiplying by a constant, we can clear denominators and make is so that all $\alpha_F$ are integers.
By possibly dividing by $p$ as many times as necessary, we can make it so that for at least one set $H\in \FF$, $\alpha_H$ is not a multiple of $p$.
Fix such a collection $\{\alpha_F\}_{F \in \FF}$ and set $H$.

By taking the inner product by $\vv_H$ in both sides of $\sum_{F \in \FF} \alpha_F \vv_F = \vec{0}$, we get
$$\alpha_H |H| + \sum_{\substack{F \in \FF\\ F \neq H}} \alpha_F |F \cap H| =  0  \, .$$

Taking into account that $\vv_F \cdot \vv_H = |F \cap H|$ is a multiple of $s$ for $F\neq H$ in $\FF$, the left hand side is not a multiple of $s$, a contradiction with the assumption that $\{\vv_F\}_{F\in \FF}$ is linearly dependent.
So $|\FF | \leq |E|$.

\item 
Let $\FF_k \coloneqq \{F \in \FF | \text{ $k$ does not divide $|F|$ } \}$.
Then $\FF = \FF_2 \cup \FF_3$.
Note how $\FF_2$ is a $2$-town, and $\FF_3$ is a $3$-town.
Therefore $|\FF| \leq |\FF_2| + |\FF_3| \leq 2|E|$ by the above item.


\item
If $s = \prod_i p_i^{\alpha_i}$ is the prime factorisation of $s$, then 
$$ \FF = \bigcup_i \FF_{p_i^{\alpha_i}}\, , $$
and each $\FF_{p_i^{\alpha_i}}$ is a $p_i^{\alpha_i}$-town, so if $c = \text{ number of prime factors of s }$, then
$$|\FF | \leq \sum_i |\FF_{p_i^{\alpha_i}}| \leq c |E| \, . $$
\end{enumerate}

\vspace{.8cm}

\item Reverse club problem

\begin{enumerate}
\item 
To show tightness, we give an example of a family $\FF$ with $|E|$ elements that satisfies the requirements.
Let $\FF = \binom{E}{|E| - 1}$.
In this way $A \in \FF $ has $|A| = |E| - 1$, which is even, and if $A\neq B $ are sets in $\FF$, then $A\cap B = E\setminus (A^c \cup B^c) $ has $|E| - 2$ elements, which is odd.

To show the inequality, for each set $I \subseteq E$, define the vector $\vv_I \in \F_2^E$:
$$ (\vv_I)_i =\begin{cases*}
      & 1 \text{ if $i \in I$,}\\
      & 0 \text{ otherwise.}
    \end{cases*} $$
Note that $\vv_F \cdot \vv_G = 1$ for $F\neq G$ sets in $\FF$, whereas $\vv_F \cdot \vv_F = 0$.
Furthermore, let $\mathbb{1} $ be the all-one vector.
We have $\mathbb{1} \cdot \vv_F = 0$ for any set $F \in \FF$.


Fix $T \in \FF$.
We claim that either $\{\vv_F\}_{F \in \FF}$ is linearly independent, or $\{\mathbb{1}\} \cup \{\vv_F\}_{\substack{F \in \FF \\ F \neq T}}$ is linearly independent.
In either case, the inequality $|\FF| \leq |E|$ follows.

Indeed, assume that $|\FF|$ is even, and consider a linear combination $\sum_{F\in \FF} \alpha_F \vv_F = \vec{0}$.
By taking the inner product with $\vv_G$ for some $G\in \FF$ on both sides gives us $\sum_{\substack{F\in \FF\\ F \neq G}} \alpha_F = 0$.
Fix $H \in \FF$.
Summing these equations for all $G \in\FF \setminus \{H \}$ gives us $(|\FF| - 1 ) \alpha_H +  \sum_{\substack{F \in \FF \\ F \neq H}} (|\FF| - 2 ) \alpha_F = 0$.
Note that $|\FF| = 2 $ in $\F_2$, so this gives us $\alpha_H  = 0$.
Since $H$ was generic, we conclude that $\alpha_H =0 $ for any $H \in \FF$, and $\{\vv_F\}_{F \in \FF}$ is linearly independent.

Assume now that $|\FF|$ is odd, and consider a linear combination $\alpha_0 \mathbb{1} + \sum_{\substack{F\in \FF \\ F \neq T}} \alpha_F \vv_F = \vec{0}$.
Taking the inner product with $\vv_G $ for $G\in \FF\setminus\{T\}$ gives $ \sum_{\substack{F\in \FF \\ F \neq T, F \neq G}} \alpha_F = 0$.
Fix $H \in \FF$ different from $T$.
Summing all equations for all sets $G \in \FF\setminus \{T, H\}$ gives us 
$$(|\FF| - 2) \alpha_H + \sum_{\substack{F\in \FF \\ F \neq T, F \neq H}} (|\FF| - 3) \alpha_F = 0$$
which, because $|\FF|$ is odd, gives us $\alpha_H  = 0$.
Since $H$ is generic different from $T$, we conclude that $\alpha_H = 0 $ for all $H \in \FF\setminus \{T\}$.

In this way we have $\alpha_0 \mathbb{1} = \alpha_0 \mathbb{1} + \sum_{\substack{F\in \FF \\ F \neq T}} \alpha_F \vv_F = 0$, which implies $\alpha_0 = 0$.
Thus $\{\mathbb{1}\} \cup \{\vv_F\}_{\substack{F \in \FF \\ F \neq T}}$ is linearly independent, as desired.

\item 
If $|E|$ is even, we can improve the argument above to show that $|\FF| \leq |E| - 1$.
Specifically, recall that we have that either $\{\vv_F\}_{F \in \FF}$ is linearly independent, or $\{\mathbb{1}\} \cup \{\vv_F\}_{\substack{F \in \FF \\ F \neq T}}$ is linearly independent.
Now we note that each of these sets are in fact in the vector subspace $V \subseteq F_2^n$ given by $V \{ \vv \in \F_2^n | \sum_i \vv_i = 0\}$, which has dimension $|E|-1$.
Thus, we have $|\FF| \leq \dim V = |E| - 1$.
\end{enumerate}


\item Two oddtowns

First we note that, in the case where $\mathcal B = \emptyset $ or $\mathcal C = \emptyset$, the claim follows directly.
Therefore, we can fix some $B \in \mathcal B, C \in \mathcal C$.
Furthermore, note that $\emptyset \not\in\mathcal B$ as otherwise one would have that $ 0 = |\emptyset \cap C|$ is odd, a contradiction.
Similarly, $\emptyset \not\in \mathcal C$.

For each set $I \subseteq E$, define the vector $\vv_I \in \F_2^E$:
$$ (\vv_I)_i =\begin{cases*}
      & 1 \text{ if $i \in I$,}\\
      & 0 \text{ otherwise.}
    \end{cases*} $$

Note that for $X \in \mathcal B,  Y \in \mathcal C$ we have that $\vv_X \cdot \vv_Y = 1$.
Let $Q \coloneqq \spn \{ \vv_F - \vv_B  | F \in \mathcal B \} $ and $P \coloneqq \spn \{ \vv_F - \vv_C  | F \in \mathcal C \} $.
Note that the maps $X \mapsto \vv_X - \vv_B$ and $X \mapsto \vv_X - \vv_C$ send $\mathcal B $ to $Q$ and $\mathcal C $ to $P$ injectively, thus we have $|\mathcal B| \leq |Q|$ and $|\mathcal C | \leq |P|$.
Furthermore, $P, Q$ are vector spaces over $\F_2$, thus $|Q| = 2^{\dim Q}$ and $|P| = 2^{\dim P}$.
We conclude the proof if we establish that $\dim Q + \dim P  \leq n - 1$.

Let $\vx_1, \ldots, \vx_m$ be a basis of $Q$, and define the matrix $T = \begin{pmatrix}
 \-- \-- \vx_1 \--\-- \\  \--\--  \vx_2 \-- \--\\ \vdots \\  \-- \-- \vx_m \--\-- \\  \--\--  \vv_A \--\--
\end{pmatrix}$.
Note that $\vv_B \not\in Q$: indeed, if $\vv_B = \sum_{F \in \mathcal B} \alpha_i (\vv_F - \vv_B)$, by taking the inner product by $\vv_C$ on both sides we have $1 = \sum_{F \in \mathcal B} \alpha_i (1 - 1) = 0$, a contradiction.
Therefore $\rk\, T = \dim Q + 1$.
Furthermore, $P \subseteq \ker \, T$.
Indeed, if $X \in \mathcal B,  Y \in \mathcal C$, then $(\vv_X - \vv_B)\cdot (\vv_Y - \vv_C) = \vv_X\cdot \vv_Y - \vv_X\cdot \vv_C - \vv_B\cdot \vv_Y + \vv_B\cdot \vv_C = 0$.
In addition, if $Y \in \mathcal C$, then $\vv_B\cdot (\vv_Y - \vv_C) = \vv_B\cdot \vv_Y - \vv_B\cdot \vv_C = 0$.
So $\nul\, T \geq \dim P$.
By the rank nulity theorem we conclude that $|E| = \rk \, T  + \nul \, T \leq \dim Q + 1 + \dim P$, as desired.


\item Colourings and coherent families

For each set $I \subseteq E$, define the vector $\vv_I \in \R^E$:
$$ (\vv_I)_i =\begin{cases*}
      & 1 \text{ if $i \in I$,}\\
      & 0 \text{ otherwise.}
    \end{cases*} $$
Let $\mathcal B = \{\vv_1, \ldots, \vv_m\}$ and let $T = \begin{pmatrix}
\-- \-- \vv_1 \-- \-- \\
\vdots \\
\-- \-- \vv_m \-- \-- 
\end{pmatrix}$.

Assume for sake of contradiction that $m < |E|$, so that $\rk \, T < |E|$, therefore there is some non-trivial vector $\vv \in \R^E$ such that $T \vv = 0$.
Let $I_+ = \{e \in E | \vv_e > 0\}$, $I_0 = \{e \in E | \vv_e = 0\}$ and $I_- = \{e \in E | \vv_e < 0\}$.

Let $\mathcal A \coloneqq \{ A \in \mathcal B | A \subseteq K\}$.
Because $\vv \neq 0$, there is some non-zero entry in $\vv$, so $I_0\subsetneq E$.
Therefore there exists a set $B \in \mathcal B$ such that $B \not\subseteq I_0$ (remember that $\bigcup_{A \in \mathcal B} A = E$).
That is, $\mathcal A \subsetneq \mathcal B$.
Therefore $\mathcal A$ has a two colouring $J \uplus K = E$.
We will use this colouring to construct a two colouring of $\mathcal B$, attaining a contradiction.

Specifically, we claim that if $J' = I_+ \cup (J \cap I_0) $ and $K' = I_- \cup (K \cap I_0)$, then $J' \uplus K' = E$ is a two colouring of $\mathcal B$.
We will prove each of the necessary claims to establish this: The sets $J'$ and $K'$ are disjoint, with $J' \cup K' = E$ and also that f $B \in \mathcal B$, then we have both $B \cap J' \neq \emptyset $ and $B \cap K' \neq \emptyset $.

\begin{itemize}
\item The sets $J'$ and $K'$ are disjoint sets, since $I_+, I_0, I_-$ are pairwise disjoint and $J, K$ are disjoint:
\begin{align*}
J' \cap K' =& \left( I_+ \cup (J \cap I_0) \right) \cap \left( I_- \cup (K \cap I_0) \right)\\
=& (I_+ \cap I_-) \cup (I_+ \cap K \cap I_0) \cup (J \cap I_0 \cap I_- ) \cup (J \cap I_0 \cap K \cap I_0) = \emptyset \, . 
\end{align*}

\item We indeed have $J' \cup K' = E$, as this is 
$$I_+ \cup (J \cap I_0) \cup I_- \cup (K \cap I_0) = I_+ \cup I_- \cup (I_0 \cap (J \cup K)) = I_+ \cup I_- \cup I_0 = E\, . $$

\item If $B \in \mathcal B \setminus \mathcal A$, then we have both $B \cap J' \neq \emptyset $ and $B \cap K' \neq \emptyset $.
Indeed, we have $\vv_B \cdot \vv = 0 $, so $\sum_{e\in B} \vv_e = 0$.
Because $B \subsetneq I_0$ there is some $e \in I_0$ such that $\vv_e \neq 0$.
Such index cannot be unique, because $\sum_{\substack{e\in B \\ \vv_e \neq 0}} \vv_e  = \sum_{e\in B} \vv_e = 0$ so there is some $f \in E$ such that $\vv_f\neq 0 $ and $\vv_e, \vv_f$ have opposite signs, say $\vv_f <0 $ and $\vv_e > 0$.
Therefore $e \in B \cap J'$ and $f \in B \cap K'$, so $J'\uplus K' $ is a two colouring.

\item If $B \in \mathcal A$, then we have both $B \cap J' \neq \emptyset $ and $B \cap K' \neq \emptyset $.
Indeed, by assumption we have that  $J\uplus K $ is a two-colouring of $\mathcal A$, so there are $e \in J\cap B$ and $f\in K\cap B$.
Furthermore, because $B \in \mathcal A$, $B \subseteq I_0$, so $e \in J'$ and $f \in K'$, as desired. 
\end{itemize}

The existence of a two colouring of $\mathcal B$ is a contradiction, as desired.
\end{enumerate}
\end{document}
