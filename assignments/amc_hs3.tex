\documentclass[kulak]{tplt} 
% options: kulak (default) or kul

\title{Algebraic Methods in Combinatorics}
\author{Assignment 3 - Solutions}
\date{Summer semester 2022 -- 2023}
\address{
	\textbf{Max Planck Institute for the Mathematics in the Sciences} \\
	\textbf{Universit\"at Leipzig}}


\usepackage{graphicx}
\usepackage{amssymb}
\usepackage{amsthm}
\usepackage{bbold}
\usepackage{listings}
\usepackage{lineno}
%\usepackage[margin=3cm]{geometry}
\usepackage[all,cmtip, color,matrix,arrow]{xy}
%\usepackage{amsaddr}
\usepackage{tikz-cd}
\usepackage{amsmath}%To use \text 
\usepackage[utf8]{inputenc}
\usepackage{hyperref}
\usepackage[capitalize]{cleveref}
\crefname{thm}{Theorem}{Theorems}
%\usepackage{bbold}
\usepackage[export]{adjustbox}
\usepackage{todonotes}
\usepackage{bm}
\usepackage{wrapfig}
\usepackage{float}
\usepackage{mathtools}
\usepackage{aliascnt}
\newaliascnt{eqfloat}{equation}
\newfloat{eqfloat}{h}{eqflts}
\floatname{eqfloat}{Equation}
\usepackage{dirtytalk}
\usepackage[mathscr]{euscript}


%\def\shuffle{\sqcup\mathchoice{\mkern-7mu}{\mkern-7mu}{\mkern-3.2mu}{\mkern-3.8mu}\sqcup\,}
\newcommand{\qshuffle}{\overline{\shuffle}}


\theoremstyle{definition}
\newtheorem{thm}{Theorem}[section]
\newtheorem{prop}[thm]{Proposition}
\newtheorem{lm}[thm]{Lemma}
\newtheorem{cor}[thm]{Corollary}
\newtheorem{obs}[thm]{Observation}
\newtheorem{defin}[thm]{Definition}
\newtheorem{smpl}[thm]{Example}
\newtheorem{quest}[thm]{Question}
\newtheorem{exe}[thm]{Exercise}
\newtheorem{const}[thm]{Construction}
\newtheorem{prob}[thm]{Problem}
\newtheorem{conj}[thm]{Conjecture}
\newtheorem{rem}[thm]{Remark}
\crefname{lm}{Lemma}{Lemmas}
\crefname{thm}{Theorem}{Theorems}
\crefname{prop}{Proposition}{Propositions}
\crefname{defin}{Definition}{Definitions}
\crefname{rem}{Remark}{Remarks}

\newcommand{\R}{\mathbb{R}}
\newcommand{\Z}{\mathbb{Z}}
\newcommand{\F}{\mathbb{F}}
\newcommand{\Q}{\mathbb{Q}}
\newcommand{\PP}{\mathbb{P}}

\newcommand{\one}{\mathbb{1}}

\newcommand{\CC}{\mathcal C}
\newcommand{\JJ}{\mathcal J}
\newcommand{\II}{\mathcal I}
\newcommand{\BB}{\mathcal B}
\newcommand{\FF}{\mathcal F}

%vectors
\newcommand{\vv}{\mathsf{v}}
\newcommand{\vw}{\mathsf{w}}
\newcommand{\vj}{\mathsf{j}}
\newcommand{\vx}{\mathsf{x}}
\newcommand{\vy}{\mathsf{y}}
\newcommand{\vz}{\mathsf{z}}
\newcommand{\va}{\mathsf{a}}
\newcommand{\vp}{\mathsf{p}}
\newcommand{\vc}{\mathsf{c}}

\newcommand{\spn}{\mathrm{span}}
\newcommand{\rowspn}{\mathrm{rowspan}}
\newcommand{\rk}{\mathrm{rk}}
\newcommand{\tr}{\mathrm{tr}}
\newcommand{\conv}{\mathrm{conv}}
\newcommand{\maxcut}{\mathrm{maxcut}}
\newcommand{\Tr}{\mathrm{Tr}}
\newcommand{\we}{\mathrm{we}}
\newcommand{\Id}{\mathrm{Id}}
\newcommand{\spec}{\mathrm{spec}}


\begin{document}

\maketitle
\vspace{2mm}
\begin{enumerate}
\item 
\begin{enumerate}
\item
Let $P = \{\vp_1, \ldots, \vp_m \} $ and for each $i=1, \ldots, m$ define the polynomial $F_i \in \R[x_1, \ldots, x_d] = \R[\vx]$ as
$$ F_i ( \vx ) \coloneqq ||\vx - \vp_i ||^2 - 1  = ||\vx ||^2 - 2 \vx \cdot \vp_i + ||\vp_i||^2 - 1\, .$$

The proof is concluded by showing that $\{F_i\}_{i=1}^m \cup \{ 1 \}$ is a linearly independent set, where $1$ is the constant polynomial equal to $1$.
Indeed, $F_i \in \spn \{ ||\vx||^2, x_1, \ldots, x_d, 1\}$, which is a $d+2$ dimensional space, so the linear independence implies that $m+1 \leq d+2$, as desired.


Assume that there is a linear combination $\alpha 1 + \sum_{i=1}^m \alpha_i F_i = 0 $.
By comparing the quadratic terms of this equation, we get that $\sum_{i=1}^m \alpha_i = 0$.
Note that $F_i(\vp_j) = \begin{cases} 0& \text{ if $i\neq j$,} \\ -1& \text{ if $i = j$.}\end{cases}$
By evaluation at $\vp_j $ we get 
$$ 0 = \alpha + \sum_{i=1}^m \alpha_i F_i(\vp_j) = \alpha - \alpha_j\, . $$

By summing all these equalities we get $m \alpha - \sum_{j=1}^m \alpha_j = 0$, therefore $\alpha =\frac{1}{m} \sum_{i=1}^m \alpha_i = 0$ and so $\alpha_j = \alpha = 0$.
As all coefficients must be zero, we conclude the desired linear independence.

\item 
Let $P = \{\vp_1, \ldots, \vp_m \} $ and for each $i=1, \ldots, m$ define the polynomial $F_i \in \R[x_1, \ldots, x_d] = \R[\vx]$ as
$$ F_i ( \vx ) \coloneqq \delta_1^2 - ||\vx - \vp_i ||^2  = - ||\vx ||^2 + 2 \vx \cdot \vp_i - ||\vp_i||^2 + \delta_1^2\, .$$

Define in $[m]$ the relation $i\sim j $ if $i=j$ or $||\vp_i - \vp_j || = \delta_2$.
Note that $\sim $ is reflexive and symmetric (that is $i\sim i$ and $i\sim j \Rightarrow j \sim i$).
We also have that $\sim $ is transitive.
Indeed, if $i\sim j$ and $j \sim k$, we have 
$$||\vp_i - \vp_k ||  = || \vp_i - \vp_j + \vp_j - \vp_k||  \leq ||\vp_i - \vp_j || + ||\vp_j - \vp_k|| \leq 2\delta_2 < \delta_1 \, , $$
so we must have either $i=k$ or $||\vp_i - \vp_k|| = \delta_2$, so in either case $i \sim k$.

This shows that $\sim $ is an equivalence relation on $[m]$.
Let $t$ be the number of equivalence classes, and let $I_1, \ldots, I_t $ be the corresponding equivalence classes.

The proof is concluded by showing that $\{F_i\}_{i=1}^m \cup \{ 1 \}$ is a linearly independent set, where $1$ is the constant polynomial equal to $1$.
Indeed, $F_i \in \spn \{ ||\vx||^2, x_1, \ldots, x_d, 1\}$, which is a $d+2$ dimensional space, so the linear independence implies that $m+1 \leq d+2$, as desired.

Assume that there is a linear combination $\alpha 1 + \sum_{i=1}^m \alpha_i F_i = 0 $.
By comparing the quadratic terms of this equation, we get that $\sum_{i=1}^m \alpha_i = 0$.

Note that we have 
$F_j(\vp_i) = \begin{cases} \delta_1^2 - \delta_2^2& \text{ if $i\sim j$ and $i\neq j$,} \\ \delta_1^2& \text{ if $i = j$,}\\ 0& \text{ if $i \not\sim j$.} \end{cases}$
Fix $k$ in $\{1, \ldots, t \}$, and let $i \in I_k$.
By evaluation on $\vp_i$ we get

\begin{equation}\label{eq:eq_1}
\begin{split}
0 = \alpha + \alpha_i \delta_1^2 + \sum_{\substack{j\in I_k \\ j \neq i}} \alpha_j (\delta_1^2-\delta_2^2) = \alpha + \alpha_i \delta_2^2 + \sum_{j\in I_k } \alpha_j (\delta_1^2-\delta_2^2)\, . 
\end{split}
\end{equation}

Summing all these equations for all $i\in I_k$ we get

\begin{equation}\label{eq:eq_2}
\begin{split}
0 &= |I_k|\alpha + \sum_{i\in I_k} \alpha_i \delta_2^2 + \sum_{j\in I_k } \alpha_j (\delta_1^2-\delta_2^2)|I_k| = |I_k|\alpha + \left( \sum_{i\in I_k} \alpha_i \right) \left( \delta_2^2 + (\delta_1^2-\delta_2^2)|I_k|  \right) \\
\Rightarrow \sum_{i \in I_k} \alpha_i &= \alpha \frac{-|I_k|}{\delta_2^2 + (\delta_1^2-\delta_2^2)|I_k|} \, ,
\end{split}
\end{equation}
where we note that $\delta_2^2 + (\delta_1^2-\delta_2^2)|I_k| > 0 $.

Therefore, $0 = \sum_{i=1}^m \alpha_i = \sum_{k = 1}^t \sum_{i \in I_k} \alpha_i = \alpha \sum_{k=1}^t \frac{-|I_k|}{\delta_2^2 + (\delta_1^2-\delta_2^2)|I_k|}$, which implies that $\alpha = 0$.
This applied to \eqref{eq:eq_2} gives us $\sum_{i \in I_k} \alpha_i  = 0$ for each $k = 1, \ldots, t$.
This applied to \eqref{eq:eq_1} gives us $ 0 = \delta_2^2 \alpha_ i$ for each $i = 1, \ldots, m$.
So each $\alpha_i $ vanishes, and we conclude the desired linear independence.

\item 
Let $P = \{\vp_1, \ldots, \vp_m \} $ and for each $i=1, \ldots, m$ define the polynomial $F_i \in \R[x_1, \ldots, x_d] = \R[\vx]$ as
$$ F_i ( \vx ) \coloneqq \prod_{j=1}^t \left( \delta_j^2 - ||\vx - \vp_i ||^2\right)  =  \prod_{j=1}^t \left(  - ||\vx ||^2 + 2 \vx \cdot \vp_i - ||\vp_i||^2 + \delta_j^2\right) \, .$$


The proof is concluded by showing that $\{F_i\}_{i=1}^m$ is a linearly independent set.
Indeed, 
$$F_i \in \spn \left\{ ||\vx||^{2\alpha_0} x_1^{\alpha_1} \cdots x_d^{\alpha_d}\Big| \sum_{i=0}^d \alpha_i \leq t, \, \alpha_i\in \Z_{\geq 0 } \txt{ for $i = 0 , \ldots, d $ }\right\}\, . $$
The number of sequences $\alpha = (\alpha_0, \ldots, \alpha_d)$ such that $ \sum_i \alpha_i \leq t, \, \alpha_i\in \Z_{\geq 0 }$ is $\binom{d+t}{t}$, so the linear independence implies that $m \leq \binom{d+t}{t}$, which is the desired inequality.

Assume that there is a linear combination $ \sum_{i=1}^m \alpha_i F_i = 0 $.
Note that $F_i(\vp_j ) = 0$ whenever $i \neq j$, and $F_i(\vp_i ) = \prod_{j=1}^t \delta_j^2 \neq 0$.
Therefore by evaluating the linear combination at $\vp_i $ we get that $\alpha_i =0$, and we conclude the linear independence.
\end{enumerate}

\item 
Let $k = |S|$ and $r = \lfloor \frac{k-1}{d+1} + 1 \rfloor$.
Because $k \geq (r-1)(d+1) + 1$, there is a partition of $S = S_1 \uplus \cdots \uplus S_r$ such that we can find some point $\va \in \bigcap_{i=1}^r \conv (S_i)$.

For any $\{i_0 < \cdots < i_d\} \subseteq [r]$ we have $\va \in \bigcap_{j=0}^d \conv (S_{i_j})$ so, according to the colourful Caratheodory theorem, there is a sequence $(\hat{x}_j)_{j=0}^d$ such that $\hat{x}_j\in S_{i_j} $ for each $j = 0, \ldots, d$ and $\va \in \conv (\hat{x}_j)_{j=0}^d$.

Note that for each choice $\{i_0 < \cdots < i_d\} \subseteq [r]$, the corresponding sequence $(\hat{x}_j)_{j=0}^d$ is necessarily distinct.
Furthermore, $\conv (\hat{x}_j)_{j=0}^d $ is an $S$-simplex.
Therefore, we have found that $\va $ is contained in $\binom{r}{d+1}$ distinct $S$-simplices.
Note that 
$$\binom{r}{d+1} \geq (r-d)^{d+1}\frac{1}{(d+1)!} \geq \frac{(k-1)^{d+1}}{(d+1)^{d+1} (d+1)!} \, ,$$
which is the desired polynomial bound.

\item 
For a point $\vp \in \R^2$ let $\overline{B}(\vp) = \{ \vv \in \R^2 : \, ||\vv - \vp|| \leq 1 \}$.
Note that for any two points $\vx, \vy \in \R^2$ we have $\vy \in \overline{B}(\vx) \Leftrightarrow \vx \in \overline{B}(\vy)$.

Consider the collection of convex sets $Y = \{\overline{B}(\vx)\}_{\vx \in X}$ and note that any three sets in $Y$ have a non-empty intersection.
Indeed, if $\vx_1, \vx_2, \vx_3 \in X$, then there is a unit disk $\overline{B}(\vy)$ that contains $\vx_1, \vx_2, \vx_3$.
Then $\vy \in \overline{B}(\vx_1)\cap \overline{B}(\vx_2) \cap \overline{B}(\vx_3)$, which shows that this intersection is non-empty.

Thus, Helly's theorem tells us that there is some $\vy \in \bigcap_{x \in X} \overline{B}(x) $, and this point is the centre of a unit disk that contains $X$, as desired.


\item 
Let $\FF = \{F_1, \ldots, F_m\} $
We act by induction on $m$.
For $m = d+1$, the statement is immediate.

For the induction step let $m \geq d+2$ and note that for each $i = 1, \ldots, m$ we have that $\FF \setminus  \{ F_i\} $ is a collection of $m-1\geq d+1$ sets of size $d$, where each $d+1$ sets has non-empty intersection.
Thus, by the induction hypothesis it has non empty intersection.
Let $a_i \in \bigcap_{\substack{ F \in \FF  \\ F \neq F_i }} F $.

Note that $\{a_1, \ldots, a_{m-1}\} \subseteq F_m$ and $|F_m| =d \leq m-1$ so there is some $i \neq j $ such that $a_i = a_j$.
In this case, we have $a_i \in \bigcap_{F \in \FF } F$, so this is non-empty, as desired.


\item 
It is enough to establish the result for $m = (d+1)(r-1)$.
Define $\vy_i \in \R^{d+1}$ for $i = 0, \ldots, m$ so that $(\vy_i)_j = \begin{cases}&1, \text{ if $j \in F_i$ or $j = n+1$ } \\  &0, \text{ otherwise.}\end{cases}$.
Define $\vv_i = \vec{e}_i \in\R^{r-1}$ for $i = 1, \ldots, r-1$, and define $\vv_r = - \sum_{i=1}^{r-1} \vec{e}_i$.

Define $M_i \coloneqq \{\vv_j\vy^T_i | j=1, \ldots, r \} \subseteq \R^{(r-1) \times (d+1)}$ for $i = 0, \ldots, m$.
Note that $\sum_{i=1}^r \vv_i = \vec{0}$, so $\vec{0} \in \conv \, M_i$ for all $i$.
There are $m+1$ such sets in $\R^m$, so colourful Caratheodory's theorem gives us, for each $i$, a point $\vz_i\in M_i$ and a coefficient $\alpha_i\geq 0$ such that 
\begin{equation}\label{eq:col}
 \vec{0} = \sum_{i=0}^m \alpha_i \vz_i , \quad \quad \sum_{i=0}^m \alpha_i = 1\, . 
\end{equation}

Let $f(i)$ such that $\vz_i = \vv_{f(i)}\vy_i^T$.
Multiplying by $\vec{e}_1^T$ on the left of the first equation of \eqref{eq:col} gives us 
$$ \vec{0} = \left( \sum_{i: f(i) = 1} \alpha_i \vy_i^T\right) - \left(\sum_{i: f(i) = r} \alpha_i \vy_i^T \right) \, . $$
Multiplying by $\vec{e}_1^T - \vec{e}_k^T$ on the left of the first equation of \eqref{eq:col}, for $k = 2, \ldots, r-1$ gives us 
$$ \vec{0} = \left( \sum_{i: f(i) = 1} \alpha_i \vy_i^T\right) - \left(\sum_{i: f(i) = k} \alpha_i \vy_i^T \right) \, . $$

We therefore get that $ \sum_{i: f(i) = j} \alpha_i \vy_i^T$ does not depend on $j = 1, \ldots, r$.
Recall that $\vy_i = (\vx_i, 1)$, so this gives us that $A \coloneqq  \sum_{i: f(i) = j} \alpha_i$ does not depend on $j$, so $r A = \sum_i \alpha_i = 1$ gives $A = \frac{1}{r}$.

We conclude that $\vv \coloneqq \sum_{i: f(i) = j} \frac{\alpha_i}{A}\vx_i$ is a convex combination that does not depend on $j$, that is $\va \in \conv\{\vx_i\}_{i: f(i) = j}$.

Therefore $\bigcap_j \conv\{\vx_i\}_{i: f(i) = j}\neq \emptyset$, and our desired partition was found.



\end{enumerate}
\end{document}
