\documentclass[12pt]{amsart}

\usepackage{graphicx}
\usepackage{amssymb}
\usepackage{amsthm}
\usepackage{listings}
\usepackage{lineno}
\usepackage[margin=3cm]{geometry}
\usepackage[all,cmtip, color,matrix,arrow]{xy}
\usepackage{amsaddr}
\usepackage{tikz-cd}
\usepackage{amsmath}%To use \text 
\usepackage[utf8]{inputenc}
\usepackage{hyperref}
\usepackage[capitalize]{cleveref}
\crefname{thm}{Theorem}{Theorems}
%\usepackage{bbold}
\usepackage[export]{adjustbox}
\usepackage{todonotes}
\usepackage{bm}
\usepackage{wrapfig}
\usepackage{float}
\usepackage{mathtools}
\usepackage{aliascnt}
\newaliascnt{eqfloat}{equation}
\newfloat{eqfloat}{h}{eqflts}
\floatname{eqfloat}{Equation}
\usepackage{dirtytalk}
\usepackage[mathscr]{euscript}

\newcommand*{\ORGeqfloat}{}
\let\ORGeqfloat\eqfloat
\def\eqfloat{%
  \let\ORIGINALcaption\caption
  \def\caption{%
    \addtocounter{equation}{-1}%
    \ORIGINALcaption
  }%
  \ORGeqfloat
}
\newcommand{\raul}[1]{\todo[color=green!30,inline]{Raul: #1}}


\makeatletter
\providecommand*{\shuffle}{%
  \mathbin{\mathpalette\shuffle@{}}%
}
\newcommand*{\shuffle@}[2]{%
  % #1: math style
  % #2: unused
  \sbox0{$#1\vcenter{}$}%
  \kern .15\ht0 % side bearing
  \rlap{\vrule height .25\ht0 depth 0pt width 2.5\ht0}%
  \raise.1\ht0\hbox to 2.5\ht0{%
    \vrule height 1.75\ht0 depth -.1\ht0 width .17\ht0 %
    \hfill
    \vrule height 1.75\ht0 depth -.1\ht0 width .17\ht0 %
    \hfill
    \vrule height 1.75\ht0 depth -.1\ht0 width .17\ht0 %
  }%
  \kern .15\ht0 % side bearing
}
\makeatother

%\def\shuffle{\sqcup\mathchoice{\mkern-7mu}{\mkern-7mu}{\mkern-3.2mu}{\mkern-3.8mu}\sqcup\,}
\newcommand{\qshuffle}{\overline{\shuffle}}


\theoremstyle{definition}
\newtheorem{thm}{Theorem}[section]
\newtheorem{prop}[thm]{Proposition}
\newtheorem{lm}[thm]{Lemma}
\newtheorem{cor}[thm]{Corollary}
\newtheorem{obs}[thm]{Observation}
\newtheorem{defin}[thm]{Definition}
\newtheorem{smpl}[thm]{Example}
\newtheorem{quest}[thm]{Question}
\newtheorem{prob}[thm]{Problem}
\newtheorem{conj}[thm]{Conjecture}
\newtheorem{rem}[thm]{Remark}
\crefname{lm}{Lemma}{Lemmas}
\crefname{thm}{Theorem}{Theorems}
\crefname{prop}{Proposition}{Propositions}
\crefname{defin}{Definition}{Definitions}
\crefname{rem}{Remark}{Remarks}

\newcommand{\R}{\mathbb{R}}
\newcommand{\Z}{\mathbb{Z}}


%vectors
\newcommand{\vv}{\mathsf{v}}
\newcommand{\vj}{\mathsf{j}}


\begin{document}

%% Title, authors and addresses
\title{Lecture notes on algebraic methods in combinatorics} % Subtitle

\author{Raul Penaguiao}
\email{raul.penaguiao@mis.mpg.de}
\address{Max Planck Institute Leipzig}
\keywords{}
\subjclass[2010]{}
\date{Spring semester 2023} % Date

%\begin{abstract}
%\end{abstract}

\maketitle
\section{Introduction}

\section{Preliminary definitions}

We denote the set $\{1, \dots, n\}$ by $[n]$.

\section{Combinatorics}

\subsubsection*{Oddtown}


\section{Geometry}

\subsubsection*{Joints Problem}

A \textbf{joint} in a collection $\mathcal L$ of lines in $\R^3$ is an intersection of at least three non-coplanar lines.


\begin{prop}
Given $N$ lines in $\R^2$ forming $J$ joints we have that
$$ J \leq C N^{3/_2}\, ,$$
furthermore, this bound is tight.
\end{prop}


\begin{proof}
First we show that this is indeed tight.
Incidentally, take an integer $n$, we can find $3n^2$ lines that intersect in $n^3$ joints. For $n=3$ we can see the example in \cref{fig:joints}.
The general construction is as follows: we take the collection of $n^2$ lines with direction $(0, 0, 1)$ that go through $(a, b, 0)$, where $a, b\in [n]$, and take it together with two rotations of $120^o$ and $240^o$ of this set along the axis $\{x = y = z\}$.

\begin{figure}[h]
\includegraphics[scale=.1]{../imgs/ina.png}%../imgs/joints
\caption{A construction of a collection of lines with high number of joints.\label{fig:joints}}
\end{figure}

Now, we establish the inequality for $C = 3^{3/_2}$.
First, we show that for any collection of lines $\mathcal L$, there is a line with at most $3 J^{1/_3}$ joints.
Acting by contradiction, assume otherwise and consider a polynomial $p \in \R[x, y, z]$ such that 
\begin{enumerate}
\item It is non-zero;

\item It vanishes at all joints;

\item It has degree at most $J^{1/_3}$ on each variable.
\end{enumerate}

Because a polynomial satisfying item 3 can be written as a combination of $(\lfloor J^{1/_3}\rfloor + 1)^3$ monomials, item 2 amounts to $J$ linear equations, such a non-zero polynomial exists.
We pick $p$ that minimizes the degree, and consider the polynomial
$$q \coloneqq \left(\frac{\partial}{\partial x} + \frac{\partial}{\partial y} + \frac{\partial}{\partial z} \right) p \, .$$

The polynomial $p$ restricted to any line is a polynomial of degree at most $3J^{1/_3}$.
Because each line has, by contradiction hypothesis, at least $3 J^{1/_3} + 1$ joints, this polynomial must be identically zero in each line.

It follows that all directional derivatives of $p$ at each joint $\vj$ vanish, so $q(\vj) = 0$.
Furthermore, $q$ also satisfies item 3 and has smaller degree than $p$.
By minimality, we must have $q \equiv 0$, which means that 
$$ p(x, y, zz ) = ax +by + cz \, ,$$
for some coefficients $a, b, c$.


We conclude the proof by acting on induction. For $N = 3$ we necessarily have $J\leq 1 \leq 3^{3/_2} 3^{3/_2}$.
No consider a collection of lines $\mathcal L$, and find the one line $\ell \in \mathcal L$ that has at most $3 J^{1/_3}$ joints.
By induction hypothesis, there are at most $3^{3/_2} (N-1)^{3/_2}$ joints in $\mathcal L \setminus \{\ell \}$, so 
$$ J \leq 3 J^{1/_3} + 3^{3/_2} (N-1)^{3/_2}\, , $$
as desired.
\end{proof}

\subsection*{Aknowledgments}
The author is supported by the Max Planck institute for the sciences. 
These notes are based on a lecture by Benny Sudakov at ETH, in 2014.

\bibliographystyle{alpha}
\bibliography{bibli}



\end{document}
